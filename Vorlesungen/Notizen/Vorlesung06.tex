% Options for packages loaded elsewhere
\PassOptionsToPackage{unicode}{hyperref}
\PassOptionsToPackage{hyphens}{url}
%
\documentclass[
]{article}
\usepackage{amsmath,amssymb}
\usepackage{iftex}
\ifPDFTeX
  \usepackage[T1]{fontenc}
  \usepackage[utf8]{inputenc}
  \usepackage{textcomp} % provide euro and other symbols
\else % if luatex or xetex
  \usepackage{unicode-math} % this also loads fontspec
  \defaultfontfeatures{Scale=MatchLowercase}
  \defaultfontfeatures[\rmfamily]{Ligatures=TeX,Scale=1}
\fi
\usepackage{lmodern}
\ifPDFTeX\else
  % xetex/luatex font selection
\fi
% Use upquote if available, for straight quotes in verbatim environments
\IfFileExists{upquote.sty}{\usepackage{upquote}}{}
\IfFileExists{microtype.sty}{% use microtype if available
  \usepackage[]{microtype}
  \UseMicrotypeSet[protrusion]{basicmath} % disable protrusion for tt fonts
}{}
\makeatletter
\@ifundefined{KOMAClassName}{% if non-KOMA class
  \IfFileExists{parskip.sty}{%
    \usepackage{parskip}
  }{% else
    \setlength{\parindent}{0pt}
    \setlength{\parskip}{6pt plus 2pt minus 1pt}}
}{% if KOMA class
  \KOMAoptions{parskip=half}}
\makeatother
\usepackage{xcolor}
\usepackage{color}
\usepackage{fancyvrb}
\newcommand{\VerbBar}{|}
\newcommand{\VERB}{\Verb[commandchars=\\\{\}]}
\DefineVerbatimEnvironment{Highlighting}{Verbatim}{commandchars=\\\{\}}
% Add ',fontsize=\small' for more characters per line
\newenvironment{Shaded}{}{}
\newcommand{\AlertTok}[1]{\textcolor[rgb]{1.00,0.00,0.00}{\textbf{#1}}}
\newcommand{\AnnotationTok}[1]{\textcolor[rgb]{0.38,0.63,0.69}{\textbf{\textit{#1}}}}
\newcommand{\AttributeTok}[1]{\textcolor[rgb]{0.49,0.56,0.16}{#1}}
\newcommand{\BaseNTok}[1]{\textcolor[rgb]{0.25,0.63,0.44}{#1}}
\newcommand{\BuiltInTok}[1]{\textcolor[rgb]{0.00,0.50,0.00}{#1}}
\newcommand{\CharTok}[1]{\textcolor[rgb]{0.25,0.44,0.63}{#1}}
\newcommand{\CommentTok}[1]{\textcolor[rgb]{0.38,0.63,0.69}{\textit{#1}}}
\newcommand{\CommentVarTok}[1]{\textcolor[rgb]{0.38,0.63,0.69}{\textbf{\textit{#1}}}}
\newcommand{\ConstantTok}[1]{\textcolor[rgb]{0.53,0.00,0.00}{#1}}
\newcommand{\ControlFlowTok}[1]{\textcolor[rgb]{0.00,0.44,0.13}{\textbf{#1}}}
\newcommand{\DataTypeTok}[1]{\textcolor[rgb]{0.56,0.13,0.00}{#1}}
\newcommand{\DecValTok}[1]{\textcolor[rgb]{0.25,0.63,0.44}{#1}}
\newcommand{\DocumentationTok}[1]{\textcolor[rgb]{0.73,0.13,0.13}{\textit{#1}}}
\newcommand{\ErrorTok}[1]{\textcolor[rgb]{1.00,0.00,0.00}{\textbf{#1}}}
\newcommand{\ExtensionTok}[1]{#1}
\newcommand{\FloatTok}[1]{\textcolor[rgb]{0.25,0.63,0.44}{#1}}
\newcommand{\FunctionTok}[1]{\textcolor[rgb]{0.02,0.16,0.49}{#1}}
\newcommand{\ImportTok}[1]{\textcolor[rgb]{0.00,0.50,0.00}{\textbf{#1}}}
\newcommand{\InformationTok}[1]{\textcolor[rgb]{0.38,0.63,0.69}{\textbf{\textit{#1}}}}
\newcommand{\KeywordTok}[1]{\textcolor[rgb]{0.00,0.44,0.13}{\textbf{#1}}}
\newcommand{\NormalTok}[1]{#1}
\newcommand{\OperatorTok}[1]{\textcolor[rgb]{0.40,0.40,0.40}{#1}}
\newcommand{\OtherTok}[1]{\textcolor[rgb]{0.00,0.44,0.13}{#1}}
\newcommand{\PreprocessorTok}[1]{\textcolor[rgb]{0.74,0.48,0.00}{#1}}
\newcommand{\RegionMarkerTok}[1]{#1}
\newcommand{\SpecialCharTok}[1]{\textcolor[rgb]{0.25,0.44,0.63}{#1}}
\newcommand{\SpecialStringTok}[1]{\textcolor[rgb]{0.73,0.40,0.53}{#1}}
\newcommand{\StringTok}[1]{\textcolor[rgb]{0.25,0.44,0.63}{#1}}
\newcommand{\VariableTok}[1]{\textcolor[rgb]{0.10,0.09,0.49}{#1}}
\newcommand{\VerbatimStringTok}[1]{\textcolor[rgb]{0.25,0.44,0.63}{#1}}
\newcommand{\WarningTok}[1]{\textcolor[rgb]{0.38,0.63,0.69}{\textbf{\textit{#1}}}}
\setlength{\emergencystretch}{3em} % prevent overfull lines
\providecommand{\tightlist}{%
  \setlength{\itemsep}{0pt}\setlength{\parskip}{0pt}}
\setcounter{secnumdepth}{-\maxdimen} % remove section numbering
\ifLuaTeX
  \usepackage{selnolig}  % disable illegal ligatures
\fi
\IfFileExists{bookmark.sty}{\usepackage{bookmark}}{\usepackage{hyperref}}
\IfFileExists{xurl.sty}{\usepackage{xurl}}{} % add URL line breaks if available
\urlstyle{same}
\hypersetup{
  pdftitle={Vorlesung 06},
  hidelinks,
  pdfcreator={LaTeX via pandoc}}

\title{Vorlesung 06}
\author{}
\date{}

\begin{document}
\maketitle

\subsection{Aufgaben zu Kapitel 6}\label{aufgaben-zu-kapitel-6}

\subsubsection{Welche Ausgabe erzeugt das folgende Fragment für die drei
Fälle:}\label{welche-ausgabe-erzeugt-das-folgende-fragment-fur-die-drei-fuxe4lle}

\begin{itemize}
\tightlist
\item
  \texttt{int\ num1\ =\ 9,\ num2\ =\ 10;}
\item
  \texttt{int\ num1\ =\ 11,\ num2\ =\ 10;}
\item
  \texttt{int\ num1\ =\ 13,\ num2\ =\ 10;}
\end{itemize}

\begin{Shaded}
\begin{Highlighting}[]
\NormalTok{if (num1 \textless{} num2)}
\NormalTok{    System.out.println("rot");}
\NormalTok{else}
\NormalTok{    if ((num1 {-} 2) \textless{} num2)}
\NormalTok{        System.out.println("blau");}
\NormalTok{    else}
\NormalTok{        System.out.println("weiss");}
\NormalTok{System.out.println("gelb");}
\NormalTok{Copy}
\end{Highlighting}
\end{Shaded}

\begin{Shaded}
\begin{Highlighting}[]
\NormalTok{rot}
\NormalTok{gelb}

\NormalTok{blau}
\NormalTok{gelb}

\NormalTok{weiss}
\NormalTok{gelb}

\NormalTok{Copy}
\end{Highlighting}
\end{Shaded}

\subsubsection{Übersetzen Sie in eine
switch-Anweisung:}\label{uxfcbersetzen-sie-in-eine-switch-anweisung}

\begin{Shaded}
\begin{Highlighting}[]
\NormalTok{if (num == 1)}
\NormalTok{    c = \textquotesingle{}A\textquotesingle{};}
\NormalTok{else}
\NormalTok{    if (num == 2)}
\NormalTok{        c = \textquotesingle{}B\textquotesingle{};}
\NormalTok{    else}
\NormalTok{        if (num == 3)}
\NormalTok{            c = \textquotesingle{}C\textquotesingle{};}
\NormalTok{        else}
\NormalTok{            c = \textquotesingle{}Z\textquotesingle{};}
\NormalTok{Copy}
\end{Highlighting}
\end{Shaded}

\begin{Shaded}
\begin{Highlighting}[]
\NormalTok{switch (num) \{}
\NormalTok{    1 : c = \textquotesingle{}A\textquotesingle{}; break;}
\NormalTok{    2 : c = \textquotesingle{}B\textquotesingle{}; break;}
\NormalTok{    3 : c = \textquotesingle{}C\textquotesingle{}; break;}
\NormalTok{    default : z = \textquotesingle{}Z\textquotesingle{}; break;}
\NormalTok{\}}
\NormalTok{Copy}
\end{Highlighting}
\end{Shaded}

\subsubsection{Übersetzen in einen
Conditional}\label{uxfcbersetzen-in-einen-conditional}

\begin{Shaded}
\begin{Highlighting}[]
\NormalTok{System.out.println((val \textless{}= 100 ? "nicht " : "") + "grösser als 100.");}
\NormalTok{Copy}
\end{Highlighting}
\end{Shaded}

\subsubsection{\texorpdfstring{\texttt{do}-loops}{do-loops}}\label{do-loops}

\begin{Shaded}
\begin{Highlighting}[]
\NormalTok{int low = 0, high = 5;}
\NormalTok{do \{}
\NormalTok{    low++;}
\NormalTok{    System.out.println(low);}
\NormalTok{\} while (low \textless{} high);}
\NormalTok{Copy}
\end{Highlighting}
\end{Shaded}

\begin{Shaded}
\begin{Highlighting}[]
\NormalTok{1}
\NormalTok{2}
\NormalTok{3}
\NormalTok{4}
\NormalTok{5}
\NormalTok{Copy}
\end{Highlighting}
\end{Shaded}

\begin{Shaded}
\begin{Highlighting}[]
\NormalTok{int low = 5, high = 5;}
\NormalTok{do \{}
\NormalTok{    System.out.println(low);}
\NormalTok{    low++;}
\NormalTok{\} while (low \textless{} high);}
\NormalTok{Copy}
\end{Highlighting}
\end{Shaded}

\begin{Shaded}
\begin{Highlighting}[]
\NormalTok{5}
\NormalTok{Copy}
\end{Highlighting}
\end{Shaded}

\subsubsection{Welche Ausgaben erzeugen folgende
Schleifen?}\label{welche-ausgaben-erzeugen-folgende-schleifen}

\begin{Shaded}
\begin{Highlighting}[]
\NormalTok{int max = 6;}
\NormalTok{for (int i = 0; i \textless{} max; i++)}
\NormalTok{    System.out.println(max + i);}
\NormalTok{Copy}
\end{Highlighting}
\end{Shaded}

\begin{Shaded}
\begin{Highlighting}[]
\NormalTok{6}
\NormalTok{7}
\NormalTok{8}
\NormalTok{9}
\NormalTok{10}
\NormalTok{11}
\NormalTok{Copy}
\end{Highlighting}
\end{Shaded}

\begin{Shaded}
\begin{Highlighting}[]
\NormalTok{int val = 0;}
\NormalTok{int num;}
\NormalTok{for (num = 10; num \textless{}= 40; num+=10)}
\NormalTok{    val += num;}
\NormalTok{System.out.println(num);}
\NormalTok{System.out.println(val);}
\NormalTok{Copy}
\end{Highlighting}
\end{Shaded}

\begin{Shaded}
\begin{Highlighting}[]
\NormalTok{50}
\NormalTok{100}
\NormalTok{Copy}
\end{Highlighting}
\end{Shaded}

\begin{Shaded}
\begin{Highlighting}[]
\NormalTok{for (int i = 0; i \textless{}= 3; i++) \{}
\NormalTok{    for (int j = i + 1; j \textgreater{} 0; j{-}{-})}
\NormalTok{        System.out.print(i * j + " ");}
\NormalTok{    System.out.println();}
\NormalTok{\}}
\NormalTok{Copy}
\end{Highlighting}
\end{Shaded}

\begin{Shaded}
\begin{Highlighting}[]
\NormalTok{0}
\NormalTok{2 1}
\NormalTok{6 4 2}
\NormalTok{12 9 6 3}
\NormalTok{Copy}
\end{Highlighting}
\end{Shaded}

\subsubsection{\texorpdfstring{Machen Sie die Klasse
\texttt{Box}generisch}{Machen Sie die Klasse Boxgenerisch}}\label{machen-sie-die-klasse-boxgenerisch}

\begin{Shaded}
\begin{Highlighting}[]
\NormalTok{public class Box \{}

\NormalTok{    private String content;}
    
\NormalTok{    public Box(String content) \{}
\NormalTok{        this.content = content;}
\NormalTok{    \}}
    
\NormalTok{    public String getContent() \{}
\NormalTok{        return this.content;}
\NormalTok{    \}}
\NormalTok{\}}
\NormalTok{Copy}
\end{Highlighting}
\end{Shaded}

\begin{Shaded}
\begin{Highlighting}[]
\NormalTok{public class Box\textless{}T\textgreater{} \{}

\NormalTok{    private T content;}
    
\NormalTok{    public Box(T content) \{}
\NormalTok{        this.content = content;}
\NormalTok{    \}}
    
\NormalTok{    public T getContent() \{}
\NormalTok{        return this.content;}
\NormalTok{    \}}
\NormalTok{\}}
\NormalTok{Copy}
\end{Highlighting}
\end{Shaded}

\begin{itemize}
\tightlist
\item
  Instanziieren Sie eine Liste boxes, welche
  \texttt{Box\textless{}String\textgreater{}} Objekte aufnehmen kann.
\end{itemize}

\begin{Shaded}
\begin{Highlighting}[]
\NormalTok{ArrayList\textless{}Box\textless{}String\textgreater{}\textgreater{} boxes = new ArrayList\textless{}Box\textless{}String\textgreater{}\textgreater{};}

\NormalTok{for (Box\textless{}String\textgreater{} b :boxes)}
\NormalTok{    System.out.println(b.getContent())}
\NormalTok{Copy}
\end{Highlighting}
\end{Shaded}

\subsubsection{\texorpdfstring{Ergänzen Sie die Klasse \texttt{Pasta}
mit einer Methode \texttt{equals} und einer Methode
\texttt{compareTo}.}{Ergänzen Sie die Klasse Pasta mit einer Methode equals und einer Methode compareTo.}}\label{erguxe4nzen-sie-die-klasse-pasta-mit-einer-methode-equals-und-einer-methode-compareto.}

\begin{Shaded}
\begin{Highlighting}[]
\NormalTok{public class Pasta \{}

\NormalTok{    private String name;}
\NormalTok{    private double price;}

\NormalTok{    public Pasta(String name, double price) \{}
\NormalTok{        this.name = name;}
\NormalTok{        this.price = price;}
\NormalTok{\}}
\NormalTok{Copy}
\end{Highlighting}
\end{Shaded}

\begin{Shaded}
\begin{Highlighting}[]
\NormalTok{public class Pasta \{}

\NormalTok{    private String name;}
\NormalTok{    private double price;}

\NormalTok{    public Pasta(String name, double price) \{}
\NormalTok{        this.name = name;}
\NormalTok{        this.price = price;}

\NormalTok{    public boolean equals(Pasta other) \{}
\NormalTok{    return (this.name.equals(other.name));}
\NormalTok{    \}}

\NormalTok{    public int compareTo(Pasta other) \{}
\NormalTok{    if (this.price == other.price)}
\NormalTok{        return 0;}
\NormalTok{    else if (this.price \textless{} other.price)}
\NormalTok{            return 1;}
\NormalTok{        else}
\NormalTok{            return {-}1;}
\NormalTok{\}}
\NormalTok{Copy}
\end{Highlighting}
\end{Shaded}


\end{document}
